\documentclass{article}
\usepackage[utf8]{inputenc}
\usepackage[english]{babel}
\usepackage{amsthm}
\usepackage{amsmath}
\usepackage{amssymb}
\usepackage{amsfonts}
\usepackage{gensymb}
\renewcommand\qedsymbol{$\blacksquare$}
\usepackage[margin=0.2in,top=1in,bottom=1in]{geometry}
\usepackage{textcomp}
\usepackage{enumitem}
\usepackage{tikz}
\usepackage[version=4]{mhchem}
\usepackage{enumerate}
\usepackage{graphicx}
\usepackage{wasysym}
\usepackage[font=small,labelfont=bf]{caption}


\begin{document}

\begin{center}
    \begin{Large}
    ASTR 400B Homework 3 \\
    \end{Large}
    Austin Corgan
\end{center}

\begin{center}
    \begin{tabular}{||c|c|c|c|c|c||}
        \hline\hline 
        Galaxy Name & Halo Mass ($\times 10^{12} M_{\odot}$) & Disk Mass ($\times 10^{12} M_{\odot}$) & Bulge Mass ($\times 10^{12} M_{\odot}$) & Total Mass ($\times 10^{12} M_{\odot})$ & $f_{bar}$\\
        \hline\hline 
         Milky Way & 1.975 & 0.075 & 0.010 & 2.060 & 0.041\\
         \hline 
         M31 & 1.921 & 0.120 & 0.019 & 2.060 & 0.067\\
         \hline
         M33 & 0.187 & 0.009 & 0.000 & 0.196 & 0.046\\
         \hline Local Group & 4.083 & 0.204 & 0.029 & 4.316 & 0.054\\
         \hline\hline 
    \end{tabular}
\end{center}


\noindent\textbf{1.} The total masses of the Milky Way and M31 are the same. In each case, the dark matter halo dominates the total mass. 

\medskip

\noindent\textbf{2.} The ratio of the total stellar mass of M31 is to the total stellar mass of the Milky Way is 1.635. Therefore, M31 is expected to be more luminous. 

\medskip  

\noindent\textbf{3.} The ratio of the mass of dark matter in the Milky Way to the mass of dark matter in M31 is 1.028. Given how close to 1 this is, it is fairly surprising, it is fairly surprising given the comparatively large deviation from 1 of the stellar mass ratio. 

\medskip

\noindent\textbf{4.} As seen from the $f_{bar}$ column in the table, the Baryon fraction is lower than that in the universe by about 9 to 12 percent for the galaxies in the Local Group. From what I understand, the existence of dark matter defines a galaxy, so that there is no dark matter in the universe that is not in a galaxy. Thus, if we compare $f_{bar}$ for all the galaxies in the universe to $f_{bar}$ for the entire universe, it must necessarily be lower for all the galaxies as the rest of the universe adds gas and stars without adding any dark matter. Therefore, any galaxy in particular should be expected to have a lower $f_{bar}$ than that for the entire universe (though there may be exceptions).  

\end{document}
